\documentclass[10pt,a4paper]{article}
\usepackage[utf8]{inputenc}
\usepackage[german]{babel}
\usepackage[T1]{fontenc}
\usepackage{amsmath}
\usepackage{amsfonts}
\usepackage{amssymb}
\usepackage{graphicx}
\usepackage{hyperref}
\author{Tim Hellmuth, Nils Rosen}
\renewcommand {\familydefault}{\sfdefault}
\makeindex
\title{Formelsammlung Atomphysik}
\begin{document}
\maketitle
\shorthandoff{"}
\paragraph{Atomare Masseneinheit}
\begin{align}
1u=\frac{m(^{12}C)}{12}=\frac{12g}{12N_A}=1,66 10^{27}kg
\end{align}
\paragraph{Postulate der QM}$\,$ \\
1. Elektronen bewegen sich Strahlungslos in Quantenzuständen mit bestimmten Energien.\\
2. Die Energiewerte werden durch die Quantifizierung des Drehimpulses bestimmt.\\
\begin{align}
L_m=m\frac{h}{2\pi}=m \hslash
\end{align}
3. Übergänge zwischen Quantenzuständen(Quantensprünge) finden mittels Absorption und Emission von Photonen statt.

\begin{align}
E_{Photon} = h   \nu
\end{align}
\paragraph{Energiezustände (Bohr)}$\,$ \\
\begin{align}
E_n = -\frac{m_e e^4}{2 (4\pi \epsilon_0 \hslash)^2} \frac{Z^2}{n^2}
\end{align}
\begin{align}
E_0=13,6 eV
\end{align}
\paragraph{Boltzmann'sches Strahlungsgesetz}$\,$ \\
\begin{align}
R_T= \frac{dE}{dtdF}=\sigma T^4
\end{align}
\paragraph{Spektrale Leistung pro Fläche}$\,$ \\
\begin{align}
R_T = \int_0^1 T_T(\nu) d\nu
\end{align}
\paragraph{Rayleigh-Jeans Strahlungsformel}$\,$ \\
\begin{align}
\rho_T(\nu)d\nu=\frac{8\pi k T \nu^2}{c^3}d\nu
\end{align}
\paragraph{Planksches Strahlungsgesetz}$\,$ \\
\begin{align}
\rho_T(\nu)d\nu=\frac{8\pi k T \nu^2}{c^3(e^{\frac{h\nu}{kT}}-1)}d\nu
\end{align}
\paragraph{Bragg-Laue Bedingung}$\,$ \\
\begin{align}
n\lambda = n \frac{hc}{h\nu} = 2d sin(\Theta_n) \longrightarrow E = n \frac{hc}{2d}
\end{align}
\paragraph{Totale relativistische Energie eines Photons}$\,$ \\
\begin{align}
E=m_0 c^2 \frac{1}{\sqrt{1-(\frac{v}{c})^2}}
\end{align}
\paragraph{Energie-Impuls Gleichung}$\,$ \\
\begin{align}
E^2=c^2 p^2+(m_0c^2)^2
\end{align}
\begin{align}
m_{Photon} \Leftarrow p_{Photon} = \frac{E_{Photon}}{c}
\end{align}
\paragraph{Energieerhaltung nach Stoß mit einem Elektron}$\,$ \\
\begin{align}
E'_{Ph}=\frac{E}{1+\frac{E}{m_e c^2}(1-cos(\alpha))}
\end{align}
\paragraph{Wellenlänge}$\,$ \\
\begin{align}
\lambda'-\lambda = \frac{h}{m_e c}(1-cos(\alpha))=\lambda c (1-cos(\alpha)) 
\end{align}
\paragraph{Compton-Wellenlänge}$\,$ \\
\begin{align}
\lambda c = \frac{h}{m_e c}= 2,43 \cdot 10^{-12} m
\end{align}
\paragraph{Paarerzeugung}($E_{Ph} > 2 m_e c^2 = 1022 KeV$) \\
\begin{align}
h\nu=2m_e c^2 + K_{e^-} + K_{e^+}(+E_R)
\end{align}
\begin{align}
h\nu = \sqrt{c^2p_{e^-}^2+(m_ec^2)^2}+\sqrt{c^2p_{e^+}^2+(m_ec^2)^2}
\end{align}
\paragraph{Gesamt-Wirkungsquerschnitt für Wechselwirkungen von Photonen mit Materie}$\,$ \\
\begin{align}
W=\frac{N(\Delta x)}{I}= \frac{N \sigma}{A} = n\sigma
\end{align}
\begin{align}
N(\Delta x)=\sigma n I
\end{align}
\paragraph{Beer-Lambertschen Gesetz}$\,$ \\
\begin{align}
I(t) = I(0) e^{-\rho \sigma t}
\end{align}
\paragraph{de Broglie Wellenlänge}$\,$ \\
\begin{align}
\lambda =\frac{h}{p}
\end{align}
\paragraph{EM-Welle}$\,$ \\
\begin{align}
E(x,t)= E m sin(2\pi(\frac{x}{\lambda}-\nu t))
\end{align}
\paragraph{Heisenbergsche Unschärferelation}$\,$ \\
\begin{align}
\Delta x \Delta p_x \geqslant \frac{\hslash}{2}\\
\Delta y \Delta p_y \geqslant \frac{\hslash}{2}\\
\Delta 2 \Delta p_2 \geqslant \frac{\hslash}{2}\\
\Delta E \Delta t \geqslant \frac{\hslash}{2}
\end{align}
\paragraph{Aufenthaltswahrscheinlichkeit eines Teilchens}$\,$ \\
\begin{align}
W(x,t=T)=\langle \psi(x,T)^2 \rangle = \langle\psi_0 e^{i(kx-\omega h)}\psi * e^{i(kx-\omega h)} \rangle = \langle \psi_0^2 \rangle 
\end{align}
\paragraph{Wellenpakete}$\,$ \\
\begin{align}
\psi(x,t)=sin(2\pi(\frac{x}{\lambda} \nu t))\\
x_n = \frac{n\lambda}{2}+\nu \lambda t \; (Wellenknoten)\\
g=\frac{d\nu}{d k} \; (Gruppengeschwindigkeit)\\
k=\frac{1}{\lambda}
\end{align}
\paragraph{Materiewellen eines Teilchens}$\,$ \\
\begin{align}
k=\frac{1}{\lambda}=\frac{p}{n} \Rightarrow dk = \frac{dp}{n}\\
\nu = \frac{E}{n} = \frac{p^2}{2mn}\\
g = \frac{d\nu}{dk} = h\frac{d\nu}{dp}=h \frac{d}{dp}(\frac{p^2}{2mh})= \frac{p}{m}=v
\end{align}
Eine relativistische Rechnung ergibt für die Phasengeschwindigkeit $\omega > c $ und für die Gruppengeschwindigkeit $g=v$\\
\paragraph{Mehrere Wellen}$\,$ \\
\begin{align}
\psi = \sum_{k=9}^{15} \psi_k = \sum_{k=9}^{15} A_k cos(2\pi (kx- \nu t))
\end{align}
\paragraph{Gaussche Verteilung}$\,$ \\
\begin{align}
g(k)=g(\frac{2\pi}{\lambda})=\frac{1}{\sigma \sqrt{2\pi}} e^{\frac{-(k-k_0)^2}{2\sigma^2}}
\end{align}
\paragraph{Thomsonsches Atommodell}$\,$ \\
\begin{align}
f= -\frac{e}{4\pi \epsilon_0}(\frac{4\pi}{3} \rho a^3)\frac{1}{a^2} = -\frac{e\rho a}{3 \epsilon_0}=-ka \\
\rho = \frac{3}{4\pi R^3}Ze
\end{align}
(Kraft auf ein Elektron im Abstand $a$ zum Zentrum)\\
Die Anregung entspricht einem harm. Oszillator
\begin{align}
\nu 0 \frac{1}{2\pi} \sqrt{\frac{k}{m_e}}=\frac{1}{2\pi}\sqrt{\frac{Ze^2}{4\pi \epsilon_0 R^3 m_e}}
\end{align}
\paragraph{Rutherford-Streuung}$\,$\\
\begin{align}
\frac{1}{r}=\frac{D}{2b} (cos(\phi -1))+\frac{1}{b}sin(\phi))\; (Bahngleichung)
\end{align}
\paragraph{Rutherfordsche Streuformel}$\,$\\
\begin{align}
\frac{d\sigma}{d\Omega}= (\frac{zZe^2}{2\pi \epsilon M \nu^2})^2\frac{1}{16 sin^4(\frac{\theta}{2}}\\
R_{min} = D = \frac{2Ze^2}{2\pi \epsilon_0 M \nu^2} =26 \frac{e^2}{2\pi \epsilon E_\alpha}
\end{align}
\paragraph{Quantenmechanische Beschreibung des Wasserstoffatoms}$\;$ \\
\begin{align}
r_n = 4 \pi \epsilon \frac{n^2 \hslash^2}{mZe^2}\\
\nu_n = \frac{n\hslash}{m r_n} = \frac{Ze^2}{4\pi \epsilon_0 n \hslash}\\
h\nu_{n\rightarrow k} = (-13,6 Z (\frac{1}{n^2}-\frac{1}{k^2}))eV\\
k=\frac{1}{\lambda} =- \frac{13,6 s}{nc}(\frac{1}{n^2}-\frac{1}{k^2}) = R_\infty \frac{1}{k^2}-\frac{1}{n^2}\\
R_\infty = (\frac{1}{4\pi \epsilon_0})^2\frac{me^4}{4\pi \hslash^3 c}= 10973732 \frac{1}{m}
\end{align}
\paragraph{Wilson-Sommerfeld Quantisierungsregel}$\,$\\
\begin{align}
\oint_{Periode} p_q dq = n_qh
\end{align}
$p_q$: Das Moment mit der Koordinate $q$\\
\paragraph{Elliptische Bahnen (Sommerfeld-Modell)}$\,$\\
\begin{align}
\oint_{Periode} Ld\theta = n_\theta h \Rightarrow L= n_\theta \hslash \, n_\theta = 1,2,3...\\
\oint_{Periode} Ldr = n_r h \Rightarrow L(\frac{a}{b}-1)= n_r \hslash \, n_r = 1,2,3...\\
a_n = \frac{4\pi \epsilon_0 n^2 \hslash^2}{\mu Z e^2} \; \text{(große Halbachse)}\\
b_n= a_n \frac{n_\theta}{n}\; \text{(Kleine Halbachse)}
\end{align}
\paragraph{Feinstruktur}$\,$\\
\begin{align}
E_n = - (\frac{e^2}{4\pi\epsilon_0})^2 \frac{\mu Z^2}{2n^2\hslash^2}[1+\frac{\alpha^2 Z^2}{n}(\frac{1}{n_\theta}-\frac{3}{4n})]\\
\alpha = \frac{1}{4\pi \epsilon_0}\frac{e^2}{\hslash c}\approx \frac{1}{137} \; \text{(Feinstrukturkonstante)}
\end{align}
\paragraph{Quantenmechanische Strahlung eines Wasserstoffatoms}
\begin{align}
f_{QM}= (\frac{e^2}{4\pi\epsilon_0})^2\frac{m}{4\pi\hslash^3 n^3}(\frac{2pn}{(n-p)^2n^2}\\
Mit\; n>>p\;\; f_{QM}= (\frac{e^2}{4\pi\epsilon_0})^2\frac{2mp}{4\pi\hslash^3 n^3}
\end{align}
\paragraph{Schrödingergleichung (1-dim.)}
\begin{align}
i\hslash\frac{\partial\psi(x,t)}{\partial t}= -\frac{\hslash^2}{2m}\frac{\partial^2\psi(x,t)}{\partial x^2} + V(x,t)\psi(x,t)
\end{align}
Lösung für den Fall $V(x,t)=V_0$
\begin{align}
\psi(x,t)=cos(kx-\omega t)+ isin(kx-\omega t) = e^{i(kx-\omega t)}
\end{align}
\paragraph{Schrödingergleichung (3-dim.)}
\begin{align}
i\hslash\frac{\partial\psi(x,y,z,t)}{\partial t}= -\frac{\hslash^2}{2m} \triangle\psi(x,y,z,t)+V(x,y,z,t)\psi(x,y,z,t)
\end{align}
\paragraph{Born Interpretation der Wellenfunktion}
\begin{align}
P(x,t)=\psi^*(x,t)\psi(x,t)
\end{align}
\begin{align}
\psi^*(x,t): \text{Komplex konjugierte Wellenfunktion} \\
\int P(x,t)dx = \int \psi^*(x,t)\psi(x,t) dx = 1 \forall t
\end{align}
\paragraph{Operatordarstellung von Impuls und Energie}
\begin{align}
P=\pm i\hslash \frac{\Delta}{\Delta x}
E = i\hslash \frac{\Delta}{\Delta t}
\end{align}
\paragraph{Erwartungswert Frequenz}
\begin{align}
\overline{f(x,t)} = \int f(x,t)P(x,t)dx = \int \psi^*(x,t)f(x,t)\psi(x,t)dx
\end{align}
\paragraph{Mittlere Position eines Teilchens}
\begin{align}
\overline{x}= \int_{-\infty}^{+\infty} \psi^*(x,t) x \psi(x,t)dx
\end{align}
\paragraph{Unschärfe eines Teilchens am Ort $x$}
\begin{align}
\Delta x = \sqrt{\overline{x}^2}= \sqrt{\int_{-\infty}^{+\infty} \psi^*(x,t) x^2 \psi(x,t)dx}
\end{align}
\paragraph{Mittlere pot. Energie zur Zeit $t$}
\begin{align}
V(x,t) = \int_{-\infty}^{+\infty} \psi^*(x,t) V(x,t) \psi(x,t)dx
\end{align}
\paragraph{Mittlere Energie eines Teilchens}
\begin{align}
\overline{E}=\int_{-\infty}^{+\infty} \psi^*(x,t) (i\hslash(\frac{\partial}{\partial t})) \psi(x,t)dx =i\hslash \int_{-\infty}^{+\infty} \psi^*(x,t) \frac{\partial\psi(x,t)}{\partial t} dx
\end{align}
\paragraph{Mittlerer Impuls eines Teilchens}
\begin{align}
\overline{P}=-i\hslash \int_{-\infty}^{+\infty} \psi^*(x,t) \frac{\partial\psi(x,t)}{\partial x} dx
\end{align}
\paragraph{Zeitunabhängige Schrödingergleichung}
\begin{align}
E\psi(x)= -\frac{\hslash^2}{2m}\frac{\partial^2\psi(x)}{\partial x^2} + V(x)\psi(x)
\end{align}
Die Lösungen beschreiben die stationären Zustände eines Systems.
\begin{align}
\psi(x,t) = \psi(x)e^{-i\frac{E}{\hslash}t} \text{ (Eigenfunktion)}
\end{align}
Bedingungen:\\
$\psi$ und $\frac{d\psi(x)}{dx}$ sollen
endlich, eindeutig und kontinuierlich sein.
\paragraph{Diskrete Energiezustände}
\begin{align}
\frac{\partial^2 \psi(x)}{\partial x^2} = \frac{2m}{\hslash^2}&(\underbrace{V(x) - E)\psi(x})\\
\text{Krümmung }&\text{der Eigenfunktion am Ort }x
\end{align}
\paragraph{Beispiel: Harmonischer Oszillator}
\begin{align}
E_m>E_n\rightarrow|V(x)-E_m|>|V(x)-E_n|\rightarrow |\frac{\partial^2\psi_m}{\partial x^2}|>|\frac{\partial^2\psi_n}{\partial x^2}|
\end{align}
\paragraph{Lösungen der zeitunabhängigen Schrödingergleichung}$\,$\\
1. Wellenfunktionen unterschiedlicher Energien $E_l$ und $E_k$ sind orthogonal.
\begin{align}
\int_{-\infty}^{+\infty} \psi^*_l(x,t) \psi_k(x,t)dx= S_kl
\end{align}
2. Oft findet man degenerierte Zustände (mit gleichen $E$) mit linear unabhängigen (und orthogonalen) Wellenfunktionen vor. Es ist dann möglich auf beliebige andere orthogonale Kombinationen überzugehen.
\begin{align}
\psi_+ = \alpha \psi_1(x,t) + \beta \psi_2(x,t)\, \text{und} \,\psi_- = \beta \psi_1(x,t) - \alpha \psi_2(x,t)\\
\Rightarrow\, 1.
\end{align}
Betrachten wir ein Elektron, das dabei ist, vom Zustand $E_2$ in den Zustand $E_1$ überzugehen. Daraus folgt:
\begin{align}
\psi = \alpha \psi_1(x,t) + \beta \psi_2(x,t) \\
\psi^*\psi = \alpha^*\alpha\psi_1(x)^2 + \beta^*\beta \psi_2(x)^2 +\\ \alpha^*\beta\psi_1^*(x)\psi_2(x)e^{-\frac{i(E_2-E_1)t}{\hslash}}+\\ \alpha\beta^*\psi_2^*(x)\psi_1(x)e^{+\frac{i(E_2-E_1)t}{\hslash}}
\end{align}
\begin{align}
\underbrace{\alpha^*\beta\psi_1^*(x)\psi_2(x)e^{-\frac{i(E_2-E_1)t}{\hslash}}+ \alpha\beta^*\psi_2^*(x)\psi_1(x)e^{+\frac{i(E_2-E_1)t}{\hslash}}}\\ \text{Schwingung mit der Kreisfrequenz }\omega\\
\omega = \frac{E_2-E_1}{\hslash} \Rightarrow \hslash\nu=E_2-E_1
\end{align}
Dies entspricht dem 4. Bohrschen Postulat.
\paragraph{Beispiel: Das Freie Teilchen}
\begin{align}
V(x)= 0 \leftarrow E\psi(x) = -\frac{\hslash^2}{2m}\frac{\partial^2\psi(x)}{\partial x^2}\\
\psi(x)=A e^{\pm i \frac{\sqrt{2mE}}{\hslash}x} = Ae^{\pm ikx}\\
k=\frac{\sqrt{2mE}}{\hslash}
\end{align}
Die entsprechende normierte Wellenfunktion ist im $+$ Fall:
\begin{align}
\psi(x,t)=Ae^{i(kx-\omega t)}\\
\omega= \frac{E}{\hslash}
\end{align}
Die Laufrichtung wird aus dem mittleren Impuls berechnet:
\begin{align}
\overline{P}=-i\hslash \int \psi^*(x,t) \frac{\partial \psi(x,t)}{\partial x} dx =- i \hslash i k \int A^2 e^{-i(kx-\omega t)}e^{i(kx-\omega t)}dx = \hslash k
\end{align}
Beschrieben wird also eine auslaufende Welle. Für die $-$Fall ergibt sich ein negativer Impuls und somit eine einlaufende Welle.\\
In der Praxis ist ein Teilchen nur als frei über eine endliche Länge L zu bezeichnen. Daraus folgt die Normalisierung:
\begin{align}
\int_{-\infty}^{+\infty} \psi^*(x,t) \psi(x,t)dx=\int_0^L A^2 e^{-i(kx-\omega t)}e^{i(kx-\omega t)}dx = A^2L\\ \Rightarrow A=\frac{1}{\sqrt{L}}
\end{align}
\paragraph{Potentialstufe}$\,$\\
Fall 1: $E<V_0$
\begin{align}
\text{Region 1: }x<0 \;&E\psi(x)= -\frac{\hslash^2}{2m}\frac{\partial^2\psi(x)}{\partial x^2}\\
&\psi_1(x)=Ae^{ik_1x}+Be^{-ik_1x}\\
&k_1=\frac{\sqrt{2mE}}{\hslash}\\
\text{Region 2: }x>0 \;&E\psi(x)= -\frac{\hslash^2}{2m}\frac{\partial^2\psi(x)}{\partial x^2}+V_0\psi\\
& -\frac{\hslash^2}{2m}\frac{\partial^2\psi(x)}{\partial x^2} = (E-V_0)\psi(x)\\
&\Rightarrow \psi_2(x)=Ce^{ik_2'x}+De^{-ik_2'x}\\
&k_2'=\frac{\sqrt{2m(E-V_0)}}{\hslash}=i\frac{\sqrt{2m(V_0-E)}}{\hslash} =ik_2\\
&\psi_2(x)=Ce^{-k_2'x}+De^{k_2'x}\\
&\text{Da $k_2$ real und positiv ist, muss $D=0$ sein.}\\
&\Rightarrow \psi_2(x)=Ce^{-k_2x}
\end{align}
Kontinuitätsbedingung bei $x=0$\\
\begin{align}
\psi_1(0)=\psi_2(0) \;\Rightarrow &A+B = C\\
(\frac{d\psi_1}{dx})_{x=0}=(\frac{d\psi_2}{dx})_{x=0}\;\Rightarrow &(A-B)ik_1= -Ck_2\\
\Rightarrow &A= \frac{C}{2}(1+\frac{ik_2}{k_1})\\
&B= \frac{C}{2}(1-\frac{ik_2}{k_1})
\end{align}
Eigenfunktionen
\begin{align}
x\leq 0:\;& \psi(x)=\frac{C}{2}(1+\frac{ik_2}{k_1})e^{ik_1x}+\frac{C}{2}(1-\frac{ik_2}{k_1})e^{-ik_1x}\\
x\geq 0:\;& \psi(x)=Ce^{-k_2x}\\
\end{align}
Wellenfunktionen
\begin{align}
x\leq 0:\;& \psi(x)=\frac{C}{2}(1+\frac{ik_2}{k_1})e^{i(k_1x -\omega t)}+\frac{C}{2}(1-\frac{ik_2}{k_1})e^{-i(k_1x-\omega t)}\\
x\geq 0:\;& \psi(x)=Ce^{-k_2x-i\omega t}\\
\end{align}
Reflektionskoeffizient$\;$\\
Wahscheinlichkeitsverhältnis der reflektierten und einfallenden Wellen
\begin{align}
R=\frac{B^*B}{A^*A}=\frac{(1+\frac{ik_2}{k_1})(1-\frac{ik_2}{k_1})}{(1-\frac{ik_2}{k_1})(1+\frac{ik_2}{k_1})}=1
\end{align}
Fall 2: $E>V_0$\\
Tunneleffekt\\
Wenn die Potentialstufe nur von endlicher Dicke ist, kann ein Teilchen durch diese Stufe hindurchtunneln.
\begin{align}
\text{Region 1: }x<0 \;&E\psi(x)= -\frac{\hslash^2}{2m}\frac{\partial^2\psi(x)}{\partial x^2}\\
&\psi_1(x)=Ae^{ik_1x}+Be^{-ik_1x}\\
&k_1=\frac{\sqrt{2mE}}{\hslash}\\
\text{Region 2: }x>0 \;&E\psi(x)= -\frac{\hslash^2}{2m}\frac{\partial^2\psi(x)}{\partial x^2}+V_0\psi\\
& -\frac{\hslash^2}{2m}\frac{\partial^2\psi(x)}{\partial x^2} = (E-V_0)\psi(x)\\
&\Rightarrow \psi_2(x)=Ce^{ik_2x}+De^{-ik_2x}\\
&k_2=\frac{\sqrt{2m(E-V_0)}}{\hslash} \\
&\Rightarrow \psi_2(x)=Ce^{-k_2x}\\
&\text{Da sich das Teilchen in positiver x-Richtung bewegt}\\
&\text{muss $C=0$ sein.}\\
\rightarrow \psi_2(x)=De^{ik_2x}
\end{align}
Kontinuitätsbedingung bei $x=0$\\
\begin{align}
\psi_1(0)=\psi_2(0) \;\Rightarrow &A+B = D\\
(\frac{d\psi_1}{dx})_{x=0}=(\frac{d\psi_2}{dx})_{x=0}\;\Rightarrow &(A-B)k_1= -Dk_2\\
\Rightarrow &A= \frac{D}{2}(1+\frac{k_2}{k_1})\\
&B= \frac{D}{2}(1-\frac{k_2}{k_1})
\end{align}
Eigenfunktionen
\begin{align}
x\leq 0:\;& \psi(x)=\frac{D}{2}(1+\frac{k_2}{k_1})e^{ik_1x}+\frac{D}{2}(1-\frac{ik_2}{k_1})e^{-ik_1x}\\
x\geq 0:\;& \psi(x)=De^{-k_2x}\\
\end{align}
Reflektion
\begin{align}
R=\frac{B^*B}{A^*A}=\frac{(1-\frac{k_2}{k_1})(1-\frac{k_2}{k_1})}{(1+\frac{k_2}{k_1})(1+\frac{k_2}{k_1})}=(\frac{k_1-k_2}{k_1+k_2})^2<1
\end{align}
Ein nicht reflektiertes Teilchen ist ein transmittiertes Teilchen. Entsprechen folgt für den Transmissionskoeffizienten $T$:
\begin{align}
R+T=1 \Rightarrow T=1-R = 1-\frac{B^*B}{A^*A} = 1-(\frac{k_1-k_2}{k_1+k_2})^2 = \frac{4k_1k_2}{(k_1+k_2)^2}\\=\frac{\sqrt{E(E-V_0)}}{(\sqrt{E}+\sqrt{E-V_0})^2}=\frac{\sqrt{1-\frac{V_0}{E}}}{(1+\sqrt{1-\frac{V_0}{E}})^2}
\end{align}
\paragraph{Lösung der zeitunabhängigen Schrödingergleichung II}$\;$\\
Rechteckiger Potentialtopf\\
\begin{align}
|x|>\frac{a}{2}:\; E\psi(x)= -\frac{\hslash^2}{2m}\frac{\partial^2\psi(x)}{\partial x^2} + V_0\psi(x)\\
|x|<\frac{a}{2}:\; E\psi(x)= -\frac{\hslash^2}{2m}\frac{\partial^2\psi(x)}{\partial x^2}\\
\end{align}
Gesuchte Lösung: $E<V_0$\\
Region 1: $x<-\frac{a}{2}$\\
\begin{align}
\psi_1(x)=(E-V_0)\psi(x)=-\frac{\hslash^2}{2m}\frac{\partial^2\psi(x)}{\partial x^2}
\end{align}
Region 2: $|x|<\frac{a}{2}$\\
\begin{align}
E\psi(x)= -\frac{\hslash^2}{2m}\frac{\partial^2\psi(x)}{\partial x^2} \;\Rightarrow \psi_2= A'e^{ik_1x}+B'e^{-ik_1x}\\
k_1=\frac{\sqrt{2mE}}{\hslash}\\
\text{Wir suchen gebundene Zustände (Gleichmäßige Bewegung in }\\\text{positiver/negativer-x-Richtung)}\\
A'^*A'=B'^*B' \Rightarrow A'=\pm B'\\
\psi_2(x) = A'(e^{ik_1x}-e^{-ik_1x}) = Asin(k_1x)\\
\psi_2(x) = B'(e^{ik_1x}+e^{-ik_1x}) = Bcos(k_1x)\\
\end{align}
Für jedes E, das die Randbedingungen erfüllt gibt es den generellen Lösungsansatz:
\begin{align}
\psi_2(x)=Asin(k_1x)+Bcos(k_1x) \text{ für: } -\frac{a}{2}<x<\frac{a}{2}\\ k_1=\frac{\sqrt{2mE}}{\hslash}
\end{align}
Region 3: $x>\frac{a}{2}$\\
Wie Region 1.\\
Da $(E-V_0)$ negativ ist, lauten die Lösungen:
\begin{align}
\text{Region 1}\\
\psi_1(x)=Ce^{k_2x}+De^{-k_2x}\\
\text{Region 3}\\
\psi_3(x)=Fe^{k_2x}+Ge^{-k_2x}\\
k_2=\frac{\sqrt{2m(V_0-E)}}{\hslash}
\end{align}
Da $k_2$ real und positiv ist, muss $D=F=0$ sein.
\begin{align}
\text{Region 1}\\
\psi_1(x)=Ce^{k_2x}\\
\text{Region 3}\\
\psi_3(x)=Ge^{-k_2x}\\
k_2=\frac{\sqrt{2m(V_0-E)}}{\hslash}
\end{align}
Die Kontinuitätsbedingung bei $\pm \frac{a}{2}$ liefert:
\begin{align}
-Asin(\frac{k_1a}{r}) + Bcos(\frac{k_1a}{r})&=Ce^{-\frac{k_2a}{r}}\\
Ak_1cos(\frac{k_1a}{r}) + Bk_1sin(\frac{k_1a}{r})&=Ck_2e^{-\frac{k_2a}{r}}\\
Asin(\frac{k_1a}{r}) + Bcos(\frac{k_1a}{r})&=Ge^{-\frac{k_2a}{r}}\\
Ak_1cos(\frac{k_1a}{r}) - Bk_1sin(\frac{k_1a}{r})&=Gk_2e^{-\frac{k_2a}{r}}\\
\text{Lösungen:}\\
2Asin(\frac{k_1a}{r})&=(G-C)e^{-\frac{k_2a}{r}}\\
2Bcos(\frac{k_1a}{r})&=(G+C)e^{-\frac{k_2a}{r}}\\
2Bk_1sin(\frac{k_1a}{r})&=(G+C)k_2e^{-\frac{k_2a}{r}}\\
2Ak_1cos(\frac{k_1a}{r})&=(G-C)k_2e^{-\frac{k_2a}{r}}
\end{align}
Wenn $B$ und $G+C\neq 0\Rightarrow\; k_1tan(\frac{k_1a}{r})=k_2$\\
Wenn $A$ und $G-C\neq 0\Rightarrow\; k_1cot(\frac{k_1a}{r})=-k_2$\\
Beide Gleichungen können nicht gleichzeitig genutzt werden. Daraus ergeben sich zwei Klassen von Lösungen:
\begin{align}
A=0;G=C: \; k_1tan(\frac{k_1a}{r})&= k_2\\
B=0;G=-C:\; k_1cot(\frac{k_1a}{r})&=-k_2\\
\end{align}
Eigenfuktionen für Klasse 1 aus (3)
\begin{align}
\psi_1(x)= Bcos(\frac{k_1a}{r})e^{\frac{k_2a}{r}}e^{k_2x} \; x\leq -\frac{a}{2}\\
\psi_2(x)= Bcos(k_1x)\; -\frac{a}{2} \leq x\leq \frac{a}{2}\\
\psi_3(x)= Bcos(\frac{k_1a}{r})e^{\frac{k_2a}{r}}e^{-k_2x} \; x\geq \frac{a}{2}\\
\end{align}
Eigenfuktionen für Klasse 2 aus (3)
\begin{align}
\psi_1(x)= Asin(\frac{k_1a}{r})e^{\frac{k_2a}{r}}e^{k_2x} \; x\leq -\frac{a}{2}\\
\psi_2(x)= Asin(k_1x)\; -\frac{a}{2} \leq x\leq \frac{a}{2}\\
\psi_3(x)= Asin(\frac{k_1a}{r})e^{\frac{k_2a}{r}}e^{-k_2x} \; x\geq \frac{a}{2}\\
\end{align}
Die erlaubeten ??? in beiden Klassen aus der Lösung der transzendenten Gleichungen:\\
\begin{align}
1. k_1tan(\frac{k_1a}{r}=k2\Rightarrow\;\frac{\sqrt{2mE}}{\hslash}tan(\frac{\sqrt{2mE}}{\hslash}\frac{a}{r}=\frac{\sqrt{2m(V_0-E)}}{\hslash}\\
2. k_1cot(\frac{k_1a}{r}=-k2 \Rightarrow \; \frac{\sqrt{2mE}}{\hslash}cot(\frac{\sqrt{2mE}}{\hslash}\frac{a}{r}=-\frac{\sqrt{2m(V_0-E)}}{\hslash}\\
\text{Multiplikation mit $\frac{a}{r}$}\\
1. \epsilon tan(\epsilon)= \sqrt{\frac{\omega V_0 a^2}{2\hslash}-\epsilon^2}\\
2. \epsilon cot(\epsilon)= -\sqrt{\frac{\omega V_0 a^2}{2\hslash}-\epsilon^2}\\
\epsilon=\sqrt{\frac{\omega V_0 a^2}{2\hslash}}
\end{align}
Die Gleichungen können graphisch oder numerisch gelöst werden.
\pagebreak
\paragraph{Atome mit einem Elektron} $\,$ \\
Reduktion des vorliegenden Zweikörperproblems zu einem Einkörperproblem $\,$\\
\\
Winkelabhängige Schrödingergleichungen
\begin{align}
-m^2 \Phi(\phi]=\frac{d^2 \Phi(\varphi)}{d \varphi^2}
\end{align}
\begin{align}
l(l+1)\Theta(\theta)=-\frac{1}{sin(\theta}\frac{\partial}{\partial \theta} (sin(\theta)\frac{\partial \Theta(\theta)}{\partial \theta})+\frac{m^2}{sin^2(\theta) \Theta(\theta)}
\end{align}
\begin{align}
l(l+1)R(r)=\frac{d}{dr}(r^2 \frac{d R(r)}{d r}+\frac{2\mu}{\hslash} r^2 (E-V(r)) R(r)
\end{align}
\begin{align}
l(l+1)\text{: Seperationskonstante}
\end{align}
Obige Formeln folgen aus dem Lösungsansatz:
\begin{align}
\Psi (r,\theta , \phi )=R(r) \Theta (\theta) \Phi (\varphi)
\end{align}
Lösungen der Gleichungen in $\theta $ und $\varphi $
\begin{align}
-\frac{1}{sin(\theta)} \frac{d}{d \theta} (sin(\theta) \frac{d \Theta (\theta)}{d \theta})+(\frac{m^2}{sin^2(\theta)} l(l+1)) \Theta (\theta) =0
\end{align}
Lösungen dieser Gleichung sind assoziierte Legendre-Polynome erster Art
\begin{align}
\Theta_{lm}(z)=P_{lm} (cos(\theta))=\frac{1}{2^ll!} (1-cos^2(\theta))^{\frac{\vert m \vert}{2}} \frac{d^{l+ \vert m \vert}{2}}{d (cos(\theta))^{l+\vert m \vert }} (cos(\theta - 1))^l
\end{align}
Die Polynome sind endlich wenn:
\begin{align}
l \rightarrow m = -l, -l+1, -l+2,...+l-1, +l
\end{align}
Also ist auch l ganzzahlig. Man nennt $l$ die Bahndrehimpuls-Quantenzahl.
Da beide Lösungen bei dem Potential $V(r)$ immer zusammen benötigt werden, kombinieren wie sie zu den sog. Kugeloberflächenfunktionen (spherical harmonics)
\begin{align}
Y_{lm} (\theta , \varphi ) & = \sqrt{\frac{2 l+1}{2 \phi}} \sqrt{\frac{(l-m)!}{(l-m)!}} (-1)^m P_{lm} (cos(\theta ) e^{i m \varphi})\\
m>0\\
Y_{lm}(\theta ,\varphi )& = (-1)^m Y_{l-m} (\theta , \varphi )*\\
m<0\\
m=-l, -l+1,...,+1
\end{align}
Radiale Gleichung
\begin{align}
E R(r)=-\frac{\hslash ^2}{2 \mu} \frac{1}{r^2} \frac{d}{dr} (r^2 \frac{d R(r)}{dr})+(V(r) + \frac{\hslash ^2}{2 \mu r^2} l(l+1)) R(r)
\end{align}
\paragraph{Das effektive Potential}$\,$ \\
\begin{align}
V_{eff}(r)= V(r)+V_l(r)
\end{align}
\begin{align}
V_l(r)=\frac{\hslash ^2}{2 \mu r^2} l(l+1) \text{ Zentrifugalpotential}
\end{align}
Die abnormale Länge ist eine Folge der Unschärferelation.\\
Lösung der radialen Gleichungen für das Coulomb-Potential
\begin{align}
Substitutionen
\end{align}
\begin{align}
\rho =2 \sqrt{\frac{-2 \mu E}{\hslash ^2}} r \\
\gamma = \frac{e^2 Z \sqrt{2 \mu}}{ 8 \pi \epsilon_0 \hslash \sqrt{-E}}
\end{align}
\begin{align}
\Rightarrow \frac{1}{\rho^2} \frac{d}{df}(\rho^2 \frac{d R(r)}{d \rho})+(-\frac{1}{4} -\frac{l(l+1)}{\rho^2}+\frac{\gamma}{\rho}) R(\rho)=0
\rho
\end{align}
\begin{align}
\rho \longrightarrow +\infty : \frac{1}{\rho^2}(\rho^2 \frac{dR(\rho)}{d \rho}) = \frac{R(\rho)}{4}\\
\Rightarrow R(\rho)\longrightarrow e^{\pm \frac{\rho}{2}}
\end{align}
Nur die Lösung mit positiven Vorzeichen ist akzeptabel.\\
Ansatz:
\begin{align}
R(\rho)=F(\rho) e^{-\frac{\rho}{2}}
\end{align}
\begin{align}
\Rightarrow (\frac{d^2f(\rho)}{d \rho^2})+(\frac{2}{\rho}-1)\frac{df(\rho}{d\rho}+(-\frac{l(l+1)}{\rho^2}+\frac{\gamma -1}{\rho})F(\rho)=0
\end{align}
Nun sei
\begin{align}
F(\rho)= \rho^s \sum_{k=0}^{\infty} a_k \rho ^k
\end{align}
\begin{align}
(s\geq 0, a_0 \neq 0)
\end{align}
\begin{align}
\frac{dF(\rho)}{d\rho}= \sum_{k=0}^{\infty} (k+s) a_k \rho^{k+s-1}
\end{align}
\begin{align}
\frac{d^2F(\rho)}{d\rho^2}= \sum_{k=0}^\infty (k+s)(k+s-1) a_k \rho^{k+s-2}
\end{align}
Nun:\\
Ordnung der Potenzen\\
\begin{align}
Substitutionen
\end{align}
\begin{align}
j=k-1
\end{align}
\begin{align}
j=k (k>0)
\end{align}
\begin{align}
\Rightarrow a_{j+1}=\frac{j+l+1-n}{(j+l+1)(j+l+2)-(l(l+1)} a_j
\end{align}
\begin{align}
n=\frac{e^2 Z \sqrt{2 \mu}}{8 \pi \epsilon_0 \hslash \sqrt{-E}} \Rightarrow E=-(\frac{e^2 Z}{4 \pi \epsilon_0 \hslash})^2 \frac{\mu}{2n^2}
\end{align}
Folgende Werte der Quantenzahlen sind erlaubt:
\begin{align}
n=1,2,3...
\end{align}
\begin{align}
l=0,1,2,3,...,n-1
\end{align}
\begin{align}
m=-l,-l+1,...,l-1,l
\end{align}
Jede Energie hat $n^2$ entartete (degenerierte) Zustände\\
Schrödingers Theorie liefert sowohl die Energie als auch die entsprechenden Wellenfunktion
\begin{align}
Y_{nlm}(r,\theta,\varphi,t)=R_{nl} (r) Y_{lm} (\theta,\varphi) e^{-\frac{iEt}{\hslash}}
\end{align}
\begin{align}
R_{nl}(r)=A_{nl}F_{nl}(2\sqrt{\frac{2 \mu E}{\hslash^2}}r)(2\sqrt{\frac{-2 \mu E}{\hslash^2}}r)^le^{-\sqrt{\frac{-2 \mu E}{\hslash^2}}}r)\\
= A_{nl}F_{nl}(2\frac{Zr}{a_0 n})(2\frac{Zr}{a_0 n})^le^({-\frac{Zr}{a_0 n}})
\end{align}
\begin{align}
a_0=\frac{4 \pi \epsilon_0 \hslash^2}{\mu e^2}
\end{align}
Quantisierung des Bahndrehimpulses
\begin{align}
l=-i \hslash ((y \frac{\partial}{\partial z}-z \frac{\partial}{\partial y})e_x+(z \frac{\partial}{\partial x}-x \frac{\partial}{\partial z})e_y+(x \frac{\partial}{\partial y}-y \frac{\partial}{\partial x})e_z)
\end{align}
\begin{align}
\Rightarrow l=i\hslash ((sin(\varphi) \frac{\partial}{\partial \theta}+cot(\theta)\frac{\partial}{\partial \varphi})e_x+(-cos(\varphi) \frac{\partial}{\partial \theta} \\ +cot(\theta)sin(\varphi) \frac{\partial}{\partial \varphi})e_y +(-\frac{\partial}{\partial \varphi})e_z
\end{align}
\begin{align}
l_zY_{blm}(r,\theta,\varphi)=R_{nl}(r)l_z Y_{lm}(\theta. \varphi)m \hslash Y_{nlm} (r,\theta,\varphi)
\end{align}
\begin{align}
l=\sqrt{l_x^2+l_y^2+l_z^2}
\end{align}
Man erhält
\begin{align}
l^2 Y_{nlm}(r,\theta,\varphi)=l(l+1) \hslash^2 Y_{nlm}(r,\theta.\varphi)
\end{align}
Für die $x,y$-Komponenten erhält man nur Mittelwerte:
\begin{align}
l_x=\int Y_{nlm}(r,\theta,\varphi)^*l_x Y_{nlm}(r,\theta,\varphi)d\tau=0
\end{align}
\begin{align}
l_y=\int Y_{nlm}(r,\theta,\varphi)^*l_y Y_{nlm}(r,\theta,\varphi)d\tau=0
\end{align}
Eigenwerte
\begin{align}
H=-\frac{\hslash^2}{2 \mu}\bigtriangleup +V(r)
\end{align}
\begin{align}
l_z Y_{nlm}(r,\theta,\varphi)=f(m) Y_{nlm}(r,\theta,\varphi)
\end{align}
\begin{align}
l^2 Y_{nlm}(r,\theta,\varphi)=g(m) Y_{nlm}(r,\theta,\varphi)
\end{align}
\begin{align}
H Y_{nlm}(r,\theta,\varphi)=h(m) Y_{nlm}(r,\theta,\varphi)
\end{align}
Speziell für das Coulomb-Potential
\begin{align}
f(m)=m \hslash
\end{align}
\begin{align}
g(l)=l(l+1)\hslash^2
\end{align}
\begin{align}
h(n)=En=(\frac{e^2Z}{8 \pi \epsilon_0 \hslash}=^2
\end{align}
\paragraph{Magentisches Dipolmoment}$\,$ \\
\begin{align}
\vert \mu \vert =i A = \frac{e\nu}{2 \pi r_1}=\frac{e \nu r_1}{\alpha} \;\text{Magnetfeld eines Dipols}
\end{align}
\begin{align}
\Rightarrow =-\frac{e}{2m}L
\end{align}
Komplexe quantenmechanische Rechnung
\begin{align}
\nu=-\frac{e \hslash}{2m}\sqrt{l(l+1)}
\end{align}
Natürliche Einheit des magentischen Dipolmoments (Bohrsches Magenton)
\begin{align}
\nu_B=\frac{e \hslash}{2m}=0,579*10^{-4} \frac{eV}{T}
\end{align}
Einführung des dimensionslosen orbitalen g-Faktors $g_l=1$
\begin{align}
\nu=-g_l\nu_b \sqrt{l(l+1)}
\end{align}
Magentisches Dipolmoment im Magentfeld
\begin{align}
\frac{dL}{dt}=tau=\nu \wedge B =-\frac{g_l \nu_B}{\hslash}L \wedge B
\end{align}
Präzessionsfrequenz
\begin{align}
\omega_L=\frac{g_l \nu_B}{\hslash} B \text{Lamor-Frequenz}
\end{align}
Änderung der Energie durch Präzession
\begin{align}
\delta E= -\nu B=-g_l \nu_b m_l B
\end{align}
Aufgrund der Lorentz-Kraft wird eine Gesamtkraft ind Richrung des zunehmenden Feldes erzeugt, sodass eine Translation verursacht wird.
\begin{align}
F_z=\frac{\partial B}{\partial z} \nu_z = \frac{\partial B}{\partial z} g_l \nu_b m_l
\end{align}
Außer dem orbitalen Drehimpuls hat das Elektron einen inneren Drehimpuls, Spin genannt.
\begin{align}
S=\sqrt{s+(s+1)} \hslash
\end{align}
\begin{align}
S_z=m_s \hslash
\end{align}
Dies führt zu folgender Aufspaltung
\begin{align}
F_z=-\frac{\partial B}{\partial z} g_l \nu_B m_l - \frac{\partial B}{\partial z} g_s \nu_B m_s
\end{align}
\begin{align}
g_s= \text{Spin g-Faktor}
\end{align}
\begin{align}
s=\frac{1}{2}
\end{align}
\begin{align}
m_s=\pm \frac{1}{2}
\end{align}
Um die intrinsische Struktur der Elektronen zu berücksichtigen muss man die Wellenfunktion mit neuen Funktionen, Spinaren, erweitern
\begin{align}
Y(n,l,m_l,s,m_s)=Y_{nlme}(r,\theta,\varphi,t) \chi_{s,m_s}
\end{align}
\paragraph{Spin-Bahn Wechselwirkung}$\,$ \\
\begin{align}
\nu_s=-\frac{g_s \nu_B}{\hslash}S \text{magnetisches Moment proportional zum Spin}
\end{align}
Strom des Atomkerns:
\begin{align}
I=-\frac{Z e \mu}{2 \pi r}
\end{align}
\begin{align}
j=-Zer \; \text{(Stromdichte)}
\end{align}
Magnetfeld, verursacht durch den Strom:
\begin{align}
B=-\frac{\nu_0 Ze}{4 \pi m} \frac{L}{r^3}
\end{align}
Energieverschiebung durch die Spin-Bahn-Wechselwirkung
\begin{align}
\Delta E_{SL}=-\frac{\nu B}{2}=\frac{g_s \nu_b}{2 \hslash} \frac{\nu_0 Ze}{4 \pi m r^3} SL= \frac{1}{2m^2c^2} \frac{1}{r} \frac{dV(r)}{dr} SL
\end{align}
Neue Wellenfunktion
\begin{align}
\Psi(n,l,j,m)=\sum_{mlm_s}(lm_l,sm_s \vert j_m) Y_nlm_l (r,\theta, \varphi) \\chi_{x,m_s}
\end{align}
\paragraph{Feinstruktur}$\,$ \\
Gesamtenergie des Wasserstoffatoms (Diracsche Quantenmechanik)
\begin{align}
E=\frac{\nu e^4}{(4 \pi \epsilon_0)^2 2 \hslash^2 n^2}
\end{align}
Lamb-Verschiebung, entartete Zustände weisen sehr kleine Verschiebungen auf($\frac{1}{10}$ der Feinstruktur)
\begin{align}
\nu_j=\frac{g_j \nu_B J}{\hslash}
\end{align}
Mit der Landre-Form
\begin{align}
g_j=\frac{g_l(j(j+1)+l(l+1)-s(s+1)+j_s (j(j+1)-l/l+1)+s(s+1)}{2j(j+1)}
\end{align}
Nun wird am Atomkern durch das Elektron ein Magnetfeld erzeugt, welches im Mittel in Richtung des Gesamtdrehimpulses zeigt
\begin{align}
B_0=\bar{B_0} \frac{J}{j \hslash}
\end{align}
\begin{align}
\bar{B_0}=\int \Psi_{j:1+j}^* B_z \Psi_{j_1+1} d\tau
\end{align}
Resultierende Feinstruktur
\begin{align}
\Delta E_{HF}=-\nu_I B_0=-\frac{g_I \nu_N B_0}{\hslash^2 j}IJ
\end{align}
Kopplung von Elektronen und Kernspin
\begin{align}
\Delta E_{HF}= \frac{A}{2}(F(F+1)-I(I+1)-j(j+1)
\end{align}
\paragraph{Übergänge}$\,$ \\
Bei Übergängen unter Emission eines Photons sollen die Erhaltungsgesetze gelten.\\
Energie- und Impulserhaltung \\
Elektronenübergang $ E_i \rightarrow E_f (E_f<E_i)$
\begin{align}
\omega=\frac{E_i-E_f}{\slash} \Rightarrow H \nu = E_i-E_f
\end{align}
Korrektur aufgrund des auftretenen Compton-Effekts
\begin{align}
h \nu= E_i-E_f-\frac{(h \nu)^2}{2Mc}
\end{align}
Drehimpulserhaltung \\
Beobachtet werden elektrische Dipolübergänge
\begin{align}
d=-er
\end{align}
Mittelwert
\begin{align}
\bar{d}=-e \int \Psi_r^* \Psi d\tau
\end{align}
Amplitude des schwingenden Dipols
\begin{align}
p_{f_i}=\vert d \vert = \vert \int \Psi_f(r) e r \Psi_i(r) d \tau \vert
\end{align}
Quantenmechansiche Rechnung für ein Photon liefert:
\begin{align}
\lambda_{f_i}=\frac{16 \pi \nu^3}{2 h \epsilon_0 c^3} p_{f_i}^2
\end{align}
Parität\\
Elektrische Strahlung: $\pi=(-1)^L$ \\
Magnetische Strahlung: $\pi=(-1)^{L+1}$ \\
Paritätserhaltung:
\begin{align}
\pi_i \pi= \pi_f \Rightarrow (-1)^{l_i}(-1)^L=(-1)^{lf}
\end{align}
\begin{align}
(l_i-l_f)= \pm 1
\end{align}
\paragraph{Lebensdauer angeregter Zustände} $\,$ \\
Energieunschärfe:
\begin{align}
dE\geq \frac{\slash}{2 \tau} \\
 \tau \text{: Lebensdauer}
\end{align}
Energieverteilung in Form einer Resonanz
\begin{align}
f(E)\sim \frac{dE^2}{(E-E_0)^2+\frac{dE^2}{4}}
\end{align}
Beispiel: Die zwei Elektronen in einem Helium-Atom besitzen Wellenfunktionen welche sich absolut überlagern können. Demnach ist es unmöglich zu wissen welche sich an welchem Ort befindet.
\begin{align}
E_T=E_1+E_2 = &-\frac{1}{\Psi(x_1, y_1, z_1)} \frac{\hslash^2}{2m} \vartriangle \Psi (x_1, y_1, z_1) +V(x_1, y_1, z_1)\\ &- \frac{1}{\Psi(x_2, y_2, z_2)} \frac{\hslash^2}{2m} \vartriangle \Psi (x_2, y_2, z_2) +V(x_2, y_2, z_2)
\end{align}
Bei Mehrteilchensystemen, wobei die identischen Teilchen nicht miteinander wechselwirken, jedoch mit einem Potential, braucht man nur die Lösung des Einteilchenproblems zu betrachten:
\begin{align}
E_{\alpha} \Psi_{\alpha} (x, y, z) = -\frac{\hslash^2}{2m} \vartriangle \Psi_{\alpha} (x_1, y_1, z_1) +V(x_1, y_1, z_1) \Psi_{\alpha} (x, y, z) \\
\alpha: \text{Quantenzahlen}
\end{align}
Nun muss die wichtige Wertefunktion gefunden werden, welche die Unterscheidbarkeit beider Teilchen berücksichtigt.
\begin{align}
\Psi_T=\Psi_{\alpha} (x_1, y_1, z_1) \Psi_{\beta} (x_2, y_2, z_2) \equiv \Psi_{\alpha} (1) \Psi_{\beta} (2)
\end{align}
\begin{align}
P(1,2)=\Psi_T^* \Psi_T =\Psi_{\alpha} ^* (1) \Psi_{\beta} ^* (2) \Psi_{\alpha} (1) \Psi_{\alpha} (2)
\end{align}
\begin{align}
P(1,2) = P(2,1) \text{ Widerspruch!}
\end{align}
Stadtessen:
\begin{align}
\Psi_{\pm} (1,2)= \frac{1}{\sqrt{2}} (\Psi_{\alpha}(1) \Psi_{\beta}(2) \pm \Psi_{\alpha}(2) \Psi_{\beta}(1) \\
(\alpha \neq \beta)
\end{align}
Somit
\begin{align}
\Psi_{\pm} (1,2) = \pm \Psi_{\pm} (2,1)
\end{align}
Es resultieren zwei zueinander orthogonal (entartete) Lösungen:
\begin{align}
\int \int \Psi_{\alpha} (1,2)^* \Psi_{\alpha} (1,2) d\tau_1 d\tau_2 =0
\end{align}
\begin{align}
\Psi_+ (1,2) = \frac{1}{2} (\Psi_{\alpha} (1) \Psi_{\alpha} (2) +  \Psi_{\alpha} (2) \Psi_{\alpha} (1)) = \Psi_{\alpha} (1) \Psi_{\alpha} (2)
\end{align}
\begin{align}
\Psi_-(1,2)=  \frac{1}{2} (\Psi_{\alpha} (1) \Psi_{\alpha} (2) +  \Psi_{\alpha} (2) \Psi_{\alpha} (1)) = 0
\end{align}
Die antisymmetrische Kombination verschwindet also.
\paragraph{Pauli-Prinzip} $\,$ \\
Elektronen in Atomen können nicht zweimal einen Zustand mit denselben Quantenzahlen $ n, l, s, j, m $ besetzen. Alle Teilchen mit den halbzahligen Spin (Fermionen) können nur antisymmetrischen Kombinationen eingehen. Teilchen mit ganzzahligen Spin (Bosonen) können symmetrische Kombinationen eingehen.\\
1) Lösung der Einteilchen-Schrödingergleichung:
\begin{align}
E \Psi (x,y,z)= -\frac{hslash^2}{2m} \vartriangle \Psi (x,y,z) + V(x,y,z) \Psi (x,y,z) \Rightarrow \Psi_1 (x,y,z)
\end{align}
2)
\begin{align}
E=\sum_{k=1}^N E_k
\end{align}
3)Konstruktion der antisymmetrischen Wellenfunktionen mittles der Sluterdeterminante:
\begin{align}
\text { Matrix}
\end{align}
4) Berechnung von weiteren Observabeln und oft Störungsrechnungen der schwachen Wechselwirkung zwischen den Fermionen unter der Nutzung der Wellenfunktion als Basis.
\paragraph{Das Helium-Atom} $\,$ \\
Zunächst: Vernachlässigung der Coulomb-Abstoßung und der Feinstruktur.
\begin{align}
-\frac{\hslash^2}{2m} \vartriangle_1 \Psi_T -\frac{\hslash^2}{2m} \vartriangle_2
\end{align}
\begin{align}
E_t \Psi_T= -\frac{2 e^2}{4 \pi \epsilon_0 r_1} \Psi_T - \frac{2 e^2}{4 \pi \epsilon_0 r_2} \Psi_T
\end{align}
Einzelfunktionen
\begin{align}
E_1= \frac{\mu Z^2 e^4}{2(4 \pi \epsilon_0)^2\hslash^2 n_1^2} \\
\Psi_1 (r_1, \theta_1, \varphi_1) \chi (\frac{1}{2}, m_s)
\end{align}
\begin{align}
E_2=\frac{\mu Z^2 e^4}{2(4 \pi \epsilon_0)^2\hslash^2 n_2^2} \\
\Psi_1 (r_2, \theta_2, \varphi_2) \chi (\frac{1}{2}, m_s)
\end{align}
\begin{align}
E_T= \frac{ 4 e^4}{2(4 \pi \epsilon_0)^2\hslash^2} (\frac{1}{n_1^2}+\frac{1}{n_2^2}
\end{align}
Mögliche Gesamtspins $S$
\begin{align}
\chi_1(\frac{1}{2}, +\frac{1}{2}) \chi_2(\frac{1}{2}, +\frac{1}{2}), M_s=+1
\end{align}
\begin{align}
\chi_1(\frac{1}{2}, +\frac{1}{2}) \chi_2(\frac{1}{2}, -\frac{1}{2}), M_s=0
\end{align}
\begin{align}
\chi_1(\frac{1}{2}, -\frac{1}{2}) \chi_2(\frac{1}{2}, +\frac{1}{2}), M_s=0
\end{align}
\begin{align}
\chi_1(\frac{1}{2}, -\frac{1}{2}) \chi_2(\frac{1}{2}, -\frac{1}{2}), M_s=-1
\end{align}
Da der Spin symmetrisch und antisymmetrisch sein soll kombinieren wir die $M_s=0$-Zustände zu drei symmetrischen und einer antisymmetrischen Lösungen.
\begin{align}
\chi_1(\frac{1}{2}, +\frac{1}{2}) \chi_2(\frac{1}{2}, +\frac{1}{2}), M_s=+1
\end{align}
\begin{align}
\frac{\chi_1(\frac{1}{2}, +\frac{1}{2}) \chi_2(\frac{1}{2}, -\frac{1}{2}) + \chi_1(\frac{1}{2}, -\frac{1}{2}) \chi_2(\frac{1}{2}, +\frac{1}{2})}{\sqrt{2}} , M_s=0
\end{align}
\begin{align}
\chi_1(\frac{1}{2}, -\frac{1}{2}) \chi_2(\frac{1}{2}, -\frac{1}{2}), M_s=-1
\end{align}
\begin{align}
\frac{\chi_1(\frac{1}{2}, +\frac{1}{2}) \chi_2(\frac{1}{2}, -\frac{1}{2}) - \chi_1(\frac{1}{2}, -\frac{1}{2}) \chi_2(\frac{1}{2}, +\frac{1}{2})}{\sqrt{2}} , M_s=0
\end{align}
Grundzustand des Helium
\begin{align}
\Psi_{100} (r_1, \theta_1, \varphi_1) \Psi_{100} (r_2, \theta_2, \varphi_2) \frac{\chi_1(\frac{1}{2}, +\frac{1}{2}) \chi_2(\frac{1}{2}, -\frac{1}{2}) - \chi_1(\frac{1}{2}, -\frac{1}{2}) \chi_2(\frac{1}{2}, +\frac{1}{2})}{\sqrt{2}}
\end{align}
Lösungen sind Singuletts und Tripletts.
\paragraph{Hatree-Theorie}
\begin{align}
V(r) =\frac{e^2}{4 \pi \epsilon_0 r}
\end{align}
Für $r\longrightarrow 0$ :
\begin{align}
\frac{Ze^2}{4 \pi \epsilon_0 r}
\end{align}
Lösung der Schrödingergleichung
\begin{align}
-\frac{\hslash^2}{2m} \vartriangle \Psi + V(r) \Psi = E \Psi \Rightarrow \Psi_1(x,y,z), E_i
\end{align}
Suche des Grundzustandes durch Auffüllen der Orbitale
\begin{align}
E_G=\sum_{k=1}^Z E_K
\end{align}
Berechnung eines neuen Potentials
\begin{align}
V_{neu} (r)
\end{align}
Ladungsverteilung
\begin{align}
\rho (r) =-e \sum_{k=1}^Z \Psi_l^* (r) \Psi_k(r) \longrightarrow V_{neu}-V_{alt} < \epsilon
\end{align}
Hatree-Wellenfunktionen
\begin{align}
\Psi(n,l,m_l,s,m_s)  = R_n' l(r) Y_{lm_l}(\theta,\varphi) \chi_{s,m_s}
\end{align}
Kugeloberflächenfunktionen
\begin{align}
\sum_m Y_{ml}^* (\theta, \varphi) Y_{ml} (\theta, \varphi) = \frac{2 l+1}{4 \pi}
\end{align}
Also verhält sich jedes Orbital mit voll besetzten Unterzuständen wie eine geladene Kugel.\\
Radiale Wahrscheinlichkeitsverteilung
\begin{align}
P_{nl}(r)= 4 \pi r^2 R_{nl}^{'2} (r)
\end{align}
Mittleres Potential
\begin{align}
V(r)= \frac{Z(r) e^2}{4 \pi \epsilon_0 r}  \;Z(r)\text{ = Effektives Potential}
\end{align}
\paragraph{Ergebnisse der Hatree-Theorie}
Die numerischen Ergbnisse der Hatree-Theorie ergeben eine Fülle an einfach Regeln, die die analytischen Behandlung von wichtigen Eigenschaften von Atomen mit mehreren Elektronen erlauben.\\
1) Innere Elektronen mit $n=1$ erfahren das größte effektive Coulomb-Potential. Es ist nur durch die eigene Ladung von $-2e$ abgeschirmt. Also ist die mittlere Ladung:
\begin{align}
Z_1=Z-2
\end{align}
Aus der $Z-$ und $n-$ Abhängigkeit der Bohrschen Radien folgen für den innersten Radius:
\begin{align}
r_n=\frac{n^2}{Z}a_0 = \frac{n^2}{Z_n} a_0 \Rightarrow r_1 = \frac{a_0}{Z-2}
\end{align}
Die Elektronenorbitale werden also für ein wachsendes Z immer kleiner, speziell bei Elektronen mit $n=1$. Dies erhöht die Wechselwirkung der Elektronen mit dem Atomkern. Da die Hatree-Theorie nicht-relativistisch ist, gibt es Abweichungen für innere Elektronen für den Fall sehr schwerer Atome.\\
2) Für die inneren Elektronen kann die Energieformel von Einelektronenatome mit $Z=Z_1$ eingesetzt werden:
\begin{align}
E_n=-\frac{Z^2}{n^2} 13,6 eV \Rightarrow E_n=-(Z-2)^2 13,6 eV
\end{align}
Für Uran mit $Z=92$ ergibt dies $110 keV$ was hinsichtlich der fehlenden relativistischen Korrekturen gut mit dem experimentellen Wert von $ 116 keV$ übereinstimmt.\\
3) Etwas ganz anderes ergibt sich für die Äußersten Elektronen. Nun gilt:
\begin{align}
r_n=\frac{n^2}{Z} a_0 = \frac{n^2}{Z_n} a_0 =na_0
\end{align}
Die Radien von Atomen steigen nur sehr langsam an als Funktion von $Z$. Die Hatree-Theorie zeigt, dass diese Formel die Radien mit einem Faktor zwei überschätzt. Also gilt eher:
\begin{align}
r_n=\frac{n}{2} a_0
\end{align}
So ist z.B. das Uran-Atom mit $n=6$ nur etwas dreimal größer als ein Wasserstoffatom. Allgemein sind die meisten Atome fast gleich groß.\\
4) Die äußersten Elektronen haben Bindungsenergien, die vergleichbar sind mit den Bindungsenergien des Wasserstoffatoms:
\begin{align}
E_n=- \frac{Z_n^2}{n^2}13,6 eV =- \frac{n^2}{n^2}13,6 eV = -13,6 eV
\end{align}
Im Fall von Uran ist die Ionisationsenergie z.B. $4 eV$. Diese Eigenschaften ermöglichen es, dass sehr leichte Atome chemische Bindungen mit sehr schweren Atomen eingehen können. Dies ist wieder eine Konsequenzen des Pauli-Prinzips.\\
5) Die Hatree-Theorie zeigt im Gegenteil zur Schrödinger-Theorie eine Energieabhängigkeit von $n$ und $l$. Orbitale mit niedrigen Werten von $l$ mehr gebunden. Die gilt für die s-Orbitale:
\begin{align}
\bar{r_{nl}} =\int_0^{\infty} R_{nl} (r)^* rR_{nl} (r) 4 \pi r^2 dr
\end{align}
Die Ursache ist, dass die Entartung der $nl$-Zustände mit $ V(r) \sim \frac{1}{r} $ verbunden ist, was in Atomen mit mehreren Elektronen nicht mehr stimmt. Die Betrachtung der graphischen Darstellung in Vorlesung 16 der radialen Eigenfunktionen zeigt deutlich, dass die Wahrscheinlichkeit ein Elektron in der Nähe des Atomkerns zu finden stark abnimmt mit $l$. Da der effektive $Z$-Wert dort sehr stark ist, sind sie auch stärker gebunden.
\paragraph{Atompysikalische Beschreibung des Periodensystems} $\,$\\
Hier sind nun folgende Ergebnisse der Hatree-Theorie von Bedeutung:\\
1) Eine vollständig gefüllte Orbitale enthält $2(2l+1)$ Elektronen, die insgesamt zu einer sphätischen symmetrischen Ladunsgverteilung führt mit $ L=0$ und $J=0$.\\
2) Die Zustände mit unterschieldichen Werten mit $n$ befinden sich in unterschiedlichen Abständen $r_n$ vom Atomkern, erfahren also unterschiedliche effektive Ladungen $Z_n $ des Atomkerns.\\
3) Die Energie hängen außen von $n$ auch von $l$ ab. Je hoher $l$ , desto kleiner die Bindungsenergie.\\
4) Die für die chemischen Bindungen verantwortlichen außersten Elektronen haben für alle Atome vergleichbare Bindunsgenergien.\\
1) u. 2) bedingen, dass leichte Atome sehr stabil sind, wenn eine $n$-Schale komplett gefüllt ist (Edelgaskonfiguration). BEi schweren Atomen wird 3) wichtig, da die Aufspaltungen zwischen den $l$-Zuständen mit zunehmenden Wert größer wird. Die Hatree-Theorie liefert die bekannten Edelgaskonfigurationen.
\paragraph{Moseley-Diagramm}
Ionisierung von inneren Elektronen weist eine starke $Z$-Abhängigkeit auf.\\
Auswahlregeln für Elektronen:
\begin{align}
\vartriangle l= \pm 1
\end{align}
\begin{align}
\vartriangle j= 0, \pm 1
\end{align}
Die höchsten energetischen Linien sind proportional zu der Bindungsenergie des Grundzustandes
\begin{align}
\text{Bohr}\;\; &h \nu= Z^2 13,6 eV\\
\text{Hatree}\;\; &h \nu= (Z-2)^2 13,6 eV
\end{align}
\begin{align}
\text{k-Linie}\; \frac{1}{\lambda}=\frac{h \nu}{hc}= C(Z-a)^2 \text{ (höchste Energie)}
\end{align}
\paragraph{Optische Spektroskopie der Atome}
\begin{align}
E_n=-\frac{Z_n^2}{n^2} 13,6 eV= -\frac{n^2}{n^2} 13,6 eV = -13,6 \text{Strahlung von angeregter Atome}
\end{align}
Abweichungen von Wasserstoff sind auf die nicht-Entartung der $nl$-Zustände zurückzuführen. Die Feinstruktur durch die Spin-Bahn Wechselwirkung bewirkt dass alle Zustände außer den $s$-Zuständen Dubletts sind mit Gesamtspin:
\begin{align}
j&=l \pm \frac{1}{2} \\
l &\neq 0\\
j&= \frac{1}{2} \\
l&=0
\end{align}
Die Aufspaltung dieses Dubletts ist:
\begin{align}
\bar{E_{SL}}=\frac{\hslash^2}{4 m^2 c^2} (j(j*1)-l(l+1)-s(s+s)) \bar{\frac{1}{r}\frac{dV(r)}{dr}}
\end{align}
\begin{align}
\bar{\frac{1}{r}\frac{dV(r)}{dr}}= 4 \pi \int_0^{\infty} R_{nl}^*(r) \frac{1}{r} \frac{dV(r)}{dr} R_{nl}(r) r^2 dr
\end{align}
Die Nutzung des effektiven Hatree-Potentials $V(r) $ und der Hatree-Eigenfunktion verstärkt die Spin-Bahn Wechselwirkung mit zunehmender Ladung $Z$.
\end{document}
